\documentclass[11pt, openany]{book}
\usepackage{amsmath,amsthm,stmaryrd}
\usepackage{amsfonts}
\usepackage{amssymb}
\usepackage{mathrsfs}
\usepackage[dvips]{graphicx}
\usepackage{color}
\usepackage{tikz}
\usepackage{pstricks}
\usepackage{pifont}
\newcommand{\cmark}{\ding{51}}
\newcommand{\xmark}{\ding{55}}
\usetikzlibrary{arrows,decorations.pathmorphing,backgrounds,positioning,fit,petri,babel}
\usepackage[ruled,vlined,lined,linesnumbered,algochapter,spanish]{algorithm2e}
\usepackage[T1]{fontenc}
\usepackage{multicol}
%-----------------funcion restricci�n---------------% 
\newcommand{\restr}[1]{\raisebox{-.5ex}{$|$}_{#1}}
\newcommand{\restrd}[2]{{#1}\raisebox{-.5ex}{$|$}_{#2}}
%---------------------------------------------------%
\usepackage[scr=boondoxo,scrscaled=1.05]{mathalfa}
\usepackage[latin1]{inputenc}
\usepackage[spanish,activeacute]{babel}
\usepackage[spanish]{layout}
\usepackage{enumerate}
\usepackage{multirow,hhline}
\usepackage[hidelinks]{hyperref} 
\usepackage{array}
\usepackage[margin=2.5cm]{geometry}
\usepackage{textcomp}
\usepackage{booktabs}



%\renewcommand{\baselinestretch}{1.1}

%%%%%%%%%%%%%%%% Para margenes %%%%%%%%%%%
\hfuzz=20pt
\vfuzz=20pt
\hbadness=2000
\vbadness=\maxdimen
%-----------------------------------------%

%-------------------%DEF DE FUNCION COMPUESTA------------------%
\newcommand{\compcent}[1]{\vcenter{\hbox{$#1\circ$}}}
\newcommand{\comp}{\mathbin{\mathchoice
  {\compcent\scriptstyle}{\compcent\scriptstyle}
  {\compcent\scriptscriptstyle}{\compcent\scriptscriptstyle}}}
%-------------------%DEF DE FUNCION COMPUESTA%-----------------% 

%------------------------------nuevos comandos------------------%
\newcommand{\N}{{\ensuremath{\mathbb{N}}}}
\newcommand{\Z}{{\ensuremath{\mathbb{Z}}}}
\newcommand{\Q}{{\ensuremath{\mathbb{Q}}}}
\newcommand{\R}{{\ensuremath{\mathbb{R}}}}
\newcommand{\C}{{\ensuremath{\mathbb{C}}}}
\newcommand{\K}{{\ensuremath{\mathbb{K}}}}
\newcommand{\J}{{\ensuremath{\mathbb{J}}}}
\newcommand{\s}{{\ensuremath{\mathbb{S}}}}

\newcommand{\mca}{{\ensuremath{\mathcal{A}}}}
\newcommand{\mcb}{{\ensuremath{\mathcal{B}}}}
\newcommand{\mcc}{{\ensuremath{\mathcal{C}}}}
\newcommand{\mcs}{{\ensuremath{\mathcal{S}}}}
\newcommand{\mct}{{\ensuremath{\mathcal{T}}}}
\newcommand{\mcu}{{\ensuremath{\mathcal{U}}}}
\newcommand{\mcp}{{\ensuremath{\mathcal{P}}}}
\newcommand{\mcm}{{\ensuremath{\mathcal{M}}}}
\newcommand{\mcn}{{\ensuremath{\mathcal{N}}}}
\newcommand{\mcj}{{\ensuremath{\mathcal{J}}}}
\newcommand{\mcf}{{\ensuremath{\mathcal{F}}}}
\newcommand{\mcd}{{\ensuremath{\mathcal{D}}}}
\newcommand{\mcx}{{\ensuremath{\mathcal{X}}}}
\newcommand{\mce}{{\ensuremath{\mathcal{E}}}}

\newcommand{\scra}{{\ensuremath{\mathscr{A}}}}
\newcommand{\scrt}{{\ensuremath{\mathscr{T}}}}
\newcommand{\scru}{{\ensuremath{\mathscr{U}}}}
\newcommand{\scrp}{{\ensuremath{\mathscr{P}}}}
\newcommand{\scrv}{{\ensuremath{\mathscr{V}}}}
\newcommand{\scro}{{\ensuremath{\mathscr{O}}}}
\newcommand{\scrb}{{\ensuremath{\mathscr{B}}}}
\newcommand{\scrc}{{\ensuremath{\mathscr{C}}}}
\newcommand{\scrf}{{\ensuremath{\mathscr{F}}}}
\newcommand{\scrs}{{\ensuremath{\mathscr{S}}}}
\newcommand{\scri}{{\ensuremath{\mathscr{I}}}}
\newcommand{\scrj}{{\ensuremath{\mathscr{J}}}}
\newcommand{\scrl}{{\ensuremath{\mathscr{L}}}}
\newcommand{\scrm}{{\ensuremath{\mathscr{M}}}}
\newcommand{\scrn}{{\ensuremath{\mathscr{N}}}}
\newcommand{\scrk}{{\ensuremath{\mathscr{K}}}}
\newcommand{\scre}{{\ensuremath{\mathscr{E}}}}
\newcommand{\scrq}{{\ensuremath{\mathscr{Q}}}}

% Funci\'on salto del gradiente 
\newcommand{\lbt}{\llbracket}
\newcommand{\rbt}{\rrbracket}

% producto interior
\newcommand{\lca}{\langle}
\newcommand{\rca}{\rangle}

\theoremstyle{plain}
  \newtheorem{proposicion}{Proposici\'on}[section]
  \newtheorem{teorema}[proposicion]{Teorema}
  \newtheorem{corolario}[proposicion]{Corolario}
  \newtheorem{lema}[proposicion]{Lema}
  

\theoremstyle{definition}
  \newtheorem{definicion}[proposicion]{Definici\'on}
  \newtheorem{ejemplo}[proposicion]{Ejemplo}
  \newtheorem{ejemplos}[proposicion]{Ejemplos}
  \newtheorem{ejercicio}{Ejercicio}[chapter]
  \newtheorem{ejercicios}{Ejercicios}[chapter]
  \newtheorem{observacion}{Observaci\'on}[chapter]
  \newtheorem*{notacion}{Notaci\'on}
  

\theoremstyle{remark}
\newtheorem*{nota}{Nota}
\newtheorem{consecuencia}{Consecuencia}%[chapter]

\DeclareMathOperator{\inte}{int}
\DeclareMathOperator{\vol}{vol}
\DeclareMathOperator{\supp}{supp}
\DeclareMathOperator{\ips}{\langle \cdot, \cdot\rangle}
\DeclareMathOperator{\norm}{\|\cdot\|}
\DeclareMathOperator{\re}{Re}
\DeclareMathOperator{\id}{id}
\DeclareMathOperator{\im}{Im}
\def\card{\mathop{\rm card}}
\def\diam{\mathop{\rm diam}}
\def\dist{\mathop{\rm dist}}
\def\Lim{\mathop{\rm lim}}
\def\Inf{\mathop{\rm inf}}
\def\Max{\mathop{\rm max}}
\def\Min{\mathop{\rm min}}
\def\Liminf{\mathop{\rm lim\,inf}}
\def\Limsup{\mathop{\rm lim\,sup}}

\newcommand{\cl}{\overline}

\DeclareRobustCommand{\rchi}{{\mathpalette\irchi\relax}}
\newcommand{\irchi}[2]{\raisebox{\depth}{$#1\chi$}} % inner command, used by \rchi
 % si no queremos que a�ada la palabra "Capitulo"
%\addcontentsline{toc}{chapter}{Agradecimientos} % si queremos que aparezca en el �ndice
%\markboth{AGRADECIMIENTOS}{Agradecimiento} % encabezado 
%\addcontentsline{toc}{section}{Resumen} % si queremos que aparezca en el �ndice
%\markboth{RESUMEN}{Resumen} % encabezado
%\usepackage{nopageno}
%\usepackage{fancyhdr}
%\pagestyle{fancy}
%\pagestyle{empty}
\setlength{\headheight}{6mm}
\setlength{\parskip}{11pt}
\setlength{\parindent}{0cm}

\begin{document}
%\fancyhead{}
%\fancyhead[RE]{ \nouppercase \leftmark}
%\fancyhead[LO]{ \nouppercase \rightmark}
%\fancyhead[LE,RO]{\thepage}
%\fancyfoot{}
%\renewcommand{\headrulewidth}{0.6pt}
\bibliographystyle{plain} 

%\renewcommand{\baselinestretch}{1.35}

\makeatletter
\def\@roman#1{\romannumeral #1}
\makeatother


\thispagestyle{empty}
\vspace*{0.5cm}
\begin{center}
	{\Large\scshape Distrifull\\[5pt] }
\end{center}


\vspace*{1cm}
\begin{center}
%	{\large\scshape Avance de proyecto de grado}
\end{center}

\vspace*{1cm}
\begin{center}
	{\Large\scshape Manual de usuario}
\end{center}

\vspace*{2cm}
\begin{center}
	{\large\scshape p\'agina web: \\ [3.pt]  }
\end{center}

\vspace*{2cm}
\begin{center}
	{\rm \url{https://distrifull.herokuapp.com/} }
%	{Complejo Ruta N: Calle 67 No 52-20 Piso 1 Oficina 1079, Medell\'in, Colombia. }\\
%	\vspace*{.5mm}
%	{ 		Recepci\'on de correspondencia: Carrera 80A No 34-36. Tel\'efono 304 638 5898. }\\
%	\vspace*{.5mm}
%	{ 		Correo electr\'onico: info@asresearch.co
%		NIT: 901133054-7}\\
%	\vspace{1cm}
%	{Medell\'in}\\
\end{center}

\hfill
\begin{center}
	2\ 0\ 2\ 2
\end{center}


\pagenumbering{roman}
\frontmatter
\tableofcontents
\listoffigures
\listoftables
%\thispagestyle{empty}
\vspace*{0.5cm}
\begin{flushleft}
	{\Large\scshape \copyright Eureka infinity 2022\\[5pt] }
\end{flushleft}

\vspace*{1cm}
\begin{flushleft}
	%	{\large\scshape Avance de proyecto de grado}
\end{flushleft}

\vspace*{1cm}
\begin{flushleft}
	{\Large\scshape }
\end{flushleft}

\vspace*{2cm}
\begin{flushleft}
	{\large\scshape     }
\end{flushleft}

\vspace*{2cm}
\begin{flushleft}
	{Eureka infinity es mi proyecto personal, su finalidad es publicar todo lo relacionado con la matem\'atica que yo vaya produciendo relacionado con mi entorno,	ya se han cursos, avances de proyectos personales, hasta hoy d\'ia tengo poca experiencia profesional, pero con el tiempo, la disciplina y la constancia en el estudio, estar\'e creciendo gracias a la comunidad.
	 }
\end{flushleft}








\chapter*{Introducci\'on}
La p\'agina web de Distrifull nos permite obtener informaci\'on de productos, ponernos en contacto con la empresa, (En construcci\'on)


%\include{AGRADECIMIENTOS}

\pagestyle{plain}
\mainmatter
\chapter{Manual de usuario}
\section{Objetivo}
El objetivo de este manual es proporcionar los pasos precisos para hacer uso del aplicativo web de Distrifull. 

\section{Vista principal}
Una vez que el usuario ingrese a la direcci\'on de la p\'agina web de Distrifull, observa la siguiente vista:
\begin{figure}[h!]
	\centering
	\includegraphics[width=0.9\linewidth, height=0.3\textheight]{imagenes/inicioOne}
	\caption[Primera parte del inicio.]{Vista principal}
	\label{fig:inicioone}
\end{figure}

Esta secci\'on encontramos en la parte superior izquierda el logo de la empresa y en la parte derecha encontramos acceso directo a las secciones visibles a todo tipo de usuario, entre ellas tenemos:
\begin{description}
	\item[ ?`Qui\'enes somos?: ] este bot\'on  nos lleva a la secci\'on donde encontramos una breve explicaci\'on sobre la empresa.
	\item[Aceite digital: ] este bot\'on nos lleva a una secci\'on donde encontramos productos de la empresa.
	\item[Iniciar sesi\'on y registrarse: ] este bot\'on nos lleva a la p\'agina de inicio de sesi\'on o la p\'agina de registro.
	\item[Idioma: ] en este bot\'on podemos cambiar el idioma, en este caso tenemos el espa\~nol y el ingl\'es. 
\end{description}
%\newpage
\subsection{Registro}
Para registrarnos en la p\'agina web desde la p\'agina principal mostrada en la figura \ref{fig:inicioone} presionamos el bot\'on registrarse y nos env\'ia a la siguiente p\'agina
\begin{figure}[h!]
	\centering
	\includegraphics[width=1\linewidth, height=0.4\textheight]{imagenes/registro}
	\caption[Registro.]{Registro}
	\label{fig:inicioThree}
\end{figure}

Para realizar el registro en el aplicativo web, debemos llenar los siguientes campos: 
\begin{description}
	\item[Nombre y apellidos:] ingresar su nombre y apellidos para tener una experiencia personalizada en el sistema.
	\item[Nombre de usuario:] ingresar el nombre con que quiere ser identificado dentro de la plataforma.
	\item[Correo electr\'onico:] ingrese un correo electr\'onico al que tenga acceso, con este iniciara sesi\'on en la plataforma.
	\item[Contrase\~na:] elija una contrase\~na e ingr\'esela y tenga presente que debe recordarla para iniciar sesi\'on en la plataforma.
\end{description}
Una vez hemos culminado de llenar los campos, presionamos el bot\'on ``\textcolor{bluedistri}{CREAR CUENTA}'', apenas demos aceptar al mensaje de alerta, podemos presionar el bot\'on \textcolor{orangedistri}{iniciar sesi\'on} y accedemos a la parte interna de la p\'agina web.
\begin{figure}[h!]
	\centering
	\includegraphics[width=1\linewidth, height=0.4\textheight]{imagenes/mensajeDeRegistro}
	\caption[Mensaje de registro.]{Mensaje de confirmaci\'on de registro.}
	\label{fig:mensajederegistro}
\end{figure}

\subsection{Inicio de sesi\'on}

\begin{figure}[h!]
	\centering
	\includegraphics[width=0.8\linewidth, height=0.2\textheight]{imagenes/inicioOneTE.png}
	\caption[Primera parte del inicio.]{Bot\'on p\'agina de inicio de sesi\'on}
	\label{fig:inicioonet}
\end{figure}
Al presionar el bot\'on \textcolor{orangedistri}{{\bf iniciar sesi\'on}}, nos env\'ia a la siguiente p\'agina:
\begin{figure}[h!]
	\centering
	\includegraphics[width=0.9\linewidth, height=0.3\textheight]{imagenes/inicioSe}
	\caption[Inicio de sesi\'on.]{Inicio de sesi\'on.}
	\label{fig:iniciosesion}
\end{figure}
\newpage
Al ingresar los datos requeridos en cada campo, los cuales son: 
\begin{description}
	\item[Correo: ] En este campo ingresamos el correo con el cual hicimos el registro en la p\'agina Web.
	\item[Contrase\~na: ] En este campo ingresamos con la contrase\~na elegida al momento de realizar el registro en la plataforma.
\end{description}
Una vez tengamos todos los campos llenos, presionamos el bot\'on \textcolor{bluedistri}{{\bf\rm ENTRAR}}, si los datos son incorrectos el sistema no permite el ingreso y lanzara una alerta. Si son correctos, la 
vista que aparecer\'a depender\'a del tipo de usuario que seas, en este caso existen dos tipos de usuarios, el administrador y el no administrador.

Vista como administrador figura \ref{fig:dashboardInicio}: 


\begin{figure}[h!]
	\centering
	\includegraphics[width=1\linewidth, height=0.45\textheight]{imagenes/dashboardInicio}
	\caption[Dashboard equipo.]{Dashboard equipo}
	\label{fig:dashboardInicio}
\end{figure}

En la parte izquierda encontramos un men\'u deslizable, en la parte central encontramos la informaci\'on proporcionada por los sensores, la gr\'afica titulada {\bf Niveles de ocupaci\'on de equipos} proporciona informaci\'on sobre la capacidad del l\'iquido que tiene el tanque, la siguiente gr\'afica titulada {\bf Volumen por equipo} muestra la cantidad de l\'iquido que tiene el equipo por cierta cantidad de tiempo (cada 30 segundos), la tabla nos proporciona una parte de la informaci\'on captada por el sensor, que son fecha, hora producto, batch no (lote) y el inventario del tanque, finaliza con un \textcolor{orangedistri}{Resumen} que contiene el costo general y la capacidad de producci\'on del tanque.
\newpage
Vista sensores figura \ref{fig:dashboardSensores}: 
\begin{figure}[h!]
	\centering
	\includegraphics[width=1\linewidth, height=0.4\textheight]{imagenes/dashboardSensores}
	\caption[Dashboard sensores.]{Dashboard sensores}
	\label{fig:dashboardSensores}
\end{figure}

En esta p\'agina obtenemos: 
\begin{description}
	\item[Volumen tanque mezclado: ] indica la cantidad de producto en galones que tiene el equipo o tanque.
	\item[Estado del mezclador: ] indica si el l\'iquido est\'a en proceso de mezclado.
	\item[Estado motobomba: ] tiene tres estados, llenado nos indica que el tanque est\'a recibiendo producto, vaciado nos garantiza que el tanque est\'a evacuando producto y estable hace referencia a que se mantiene una misma cantidad de producto, es decir no sufre ning\'un cambio de volumen.
	\item[Activaci\'on motobomba: ] confirma si la moto bomba asociada al tanque est\'a encendida o apagada.
	\item[Nivel bater\'ia loraWan: ] este indicador nos permite darle seguimiento al estado de la bater\'ia, cuenta con dos estados ``Bater\'ia estable'' y ``Bater\'ia inestable''. 
	\item[Medida lineal sensor: ] indica la distancia entre el sensor y el producto en el tanque o equipo.
	\item[Activaci\'on mezclador: ] este indicador confirma si el producto o el l\'iquido en el equipo est\'a siendo mezclado, cuenta con dos estados ``APAGADO'' y ``ENCENDIDO''.
\end{description}
\newpage
Nuevos datos: 
\begin{figure}[h!]
	\centering
	\includegraphics[width=1\linewidth, height=0.4\textheight]{imagenes/dashboardEstadistica}
	\caption[Dashboard nuevos datos.]{Dashboard nuevos datos}
	\label{fig:dashboardEstadistica}
\end{figure}

En esta vista podemos actualizar el costo por gal\'on y su respectivo lote, para hacer este cambio solo llenamos el formulario que tiene los siguientes campos: 
\begin{enumerate}[{\rm 1.}]
	\item Fecha de lote.
	\item Lote.
	\item Costo unitario.
	\item Nombre del tanque.
\end{enumerate}

Para finalizar el proceso presionamos el bot\'on ``\textcolor{bluedistri}{Agregar tanque}''.

En el campo ``Elija lote y costo'' tenemos un historial donde encontramos los diferentes costos que hemos ingresado, al presionar el bot\'on ``\textcolor{bluedistri}{Cambiar}'' se actualiza la informaci\'on de costo unitario, costo general y lote en la p\'agina equipos.
\newpage
{\bf Vista como no administrador}:

\begin{figure}[h!]
	\centering
	\includegraphics[width=1\linewidth, height=0.4\textheight]{imagenes/dashboardNoAdmi}
	\caption[Dashboard no admi.]{Dashboard no administrador}
	\label{fig:registro}
\end{figure}

En la vista como no administrador, encontramos la informaci\'on proporcionada por el sensor, la cual reflejamos en la tabla de la figura \ref{fig:registro}, que tiene las siguientes columnas: 

\begin{description}
	\item[Distancia liquido sensor: ] nos indica la distancia que hay entre el sensor y el l\'iquido contenido en el tanque.
	\item[Altura del l\'iquido: ] la altura desde el inicio del tanque hasta donde llega el l\'iquido contenido en el tanque.
	\item[Volumen liquido: ] nos confirma la cantidad de l\'iquido en galones contenidos en el tanque.
	\item[Fecha llegada BD: ] indica la fecha en que la informaci\'on es proporcionada a la base de datos que almacena la informaci\'on del sensor. 
	\item[Hora llegada BD: ] indica la hora en que la informaci\'on es proporcionada a la base de datos que almacena la informaci\'on del sensor.
	\item[Fecha llegada del TTS: ] 
	\item[Hora llegada TTS: ]
	\item[Fecha llegada Gateway: ] 
	\item[Hora llegada Gateway: ]
	\item[Fecha envioSensor: ]
	\item[Hora envio sensor: ]
	\item[Estado bater\'ia: ]
\end{description}


Las otras dos vistas que son sensores y nuevos datos, se mantienen igual a las vistas del usuario administrador. 

%\newpage
\subsection{Cerrar sesi\'on}
\begin{figure}[h!]
	\centering
	\includegraphics[width=1\linewidth, height=0.3\textheight]{imagenes/cerrarSesion}
	\caption[Cerrar sesi\'on.]{Cerrar sesi\'on}
	\label{fig:cerarsesion}
\end{figure}

Para cerrar sesi\'on presionamos en el bot\'on que tiene el aspecto de tres puntos verticales, ubicado al lado del avatar del usuario, una vez lo presionamos nos aparece la opci\'on de \textcolor{bluedistri}{cerrar sesi\'on}, seleccionamos esta opci\'on y esto finaliza la sesi\'on y nos redirige a la p\'agina principal.
\newpage
\subsection{Cambio de idioma}

\begin{figure}[h!]
	\centering
	\includegraphics[width=1\linewidth, height=0.3\textheight]{imagenes/inicioOneIdioma}
	\caption[\'Area principal.]{}
	\label{fig:iniciooneidioma}
\end{figure}

El aplicativo web de Distrifull cuenta con dos idiomas, espa\~nol e ingl\'es, para cambiar el idioma de la plataforma seleccionamos el bot\'on {\bf espa\~nol} y nos aparecen las opciones para elegir el idioma, una vez seleccionemos una opci\'on, el cambio de idioma se realiza instant\'aneamente.

%\part{Bases de Datos}

\chapter{MySQL}

\section{Funciones de control de flujo}
\begin{description}
	\item[Case: ] Esta expresi\'on nos permite realizar la condici\'on y devolver el primer valor que cumpla con dicha condici\'on
		\begin{ejemplo}
			\begin{verbatim}
				
			Primer ejemplo
			
				select 
					case 1
					when 1 then 'uno'
					when 2 then 'dos'
					else 'otro n\'umero'
					end as valor;
					
			segundo ejemplo:
			
				select idFactura, idProducto,
				
				case 
					when cantidad > 2 then 'M\'as de dos productos vendidos'
					when cantidad = 2 then 'Dos productos vendidos'
					else 'Menos de dos productos vendidos'
				end as cantidad
				from detalle_factura;

			Tercer ejemplo
			
			select nombre,
			case
				when email IS NULL then 'No tiene email registrado'
				else 'email'
			end as email,
			pais
			from cliente;
			\end{verbatim}
		Podemos ver que es una sentencia muy similar a switch de vuelve el primer caso que cumpla la condici\'on. 
		\end{ejemplo}
	\item[IF :] 
	\begin{ejemplo}
		\begin{verbatim}

			Primer ejemplo
			select if(1 < 2, true, false) as resultado;
			
			segundo ejemplo
			
			select 
			idProducto,
			if(cantidad > 1, cantidad*precioUnitario, precioUnitario) as total
			from detalle_factura;
			
			Tercer ejemplo
			
			select
			nombre,
			if(fechaIngreso < '2016-12-31', concat(idEmpleado, '-16'),
				if(fechaIngreso < '2017-12-31', concat(idEmpleado, '-17'),
					if(fechaIngreso < '2018-12-31', concat(idEmpleado, '-18'),
						concat(idEmpleado,'-19')
					  )
				   )
			   ) as codigo
			from empleado;
			 
		\end{verbatim}
	\end{ejemplo}
	\item[IFNULL y NULLIF:] IFNULL nos permite evaluar una primera expresi\'on, si esta expresi\'on es null, entonces devolver\'a el segundo valor pasado por par\'ametro y NULLIF : 
	\begin{ejemplo}
		\begin{verbatim}
			
			Primer ejemplo: 
			
			select ifnull(null, 'texto') as resultado;
			
			Segundo ejemplo: 
			
			En este ejemplo devuelve los contactos de la tabla cliente en la
			columna nombre si tiene email nos da el email pero si este campo
			es null nos devuelve el tel\'efono
			
			select nombre, ifnull(email, telefono) as contacto
			from cliente;
			
			Tercer ejemplo: 
			
			select nombre,
			ifnull( (select email from cliente where idCliente = '14'), 
			'No tiene eamil registrado' )
			as email 
			from cliente
			where idCliente = '14';
			
			select 
			nullif(
				(select precioUnitario from producto where idProducto = 1),
				(select )
			)
		\end{verbatim}
	\end{ejemplo}
	
	\item[NULLIF:]

\end{description}

\section{Subqueries}
Es una declaraci\'on select en otra declaraci\'on, los subqueries devuelven datos de la consulta principal, los subqueries puede ser utilizados para agregar, actualizar, eliminar, enviar datos.

\begin{ejemplo}
	\begin{verbatim}
		
	 Ejemplo n\'umero 1: 
	 Consiste en traer cuyos empleados tengan mayor salario al promedio: 
	 
	 select idEmpleado, nombre, salario
	 from empleado
	 where salario > (select avg(salario) from empleado);

	Ejemplo 2: Seleccionamos los empleados que no pertenezcan al departamento general: 
	
	select nombre, apelllido, idDepartamento
	from empleado
	where idDepartamento NOT IN (select idDepartamento 
	                             from departamento
	                             where nombre like "%general%"
	                              );
	Ejemplo 3: facturas de los clientes que pertenezacan a Canada o Brasil:
	
	select idCliente, idFactura
	from factura
	where idCliente IN( select idCliente 
						from cliente
						where pais = 'canada' or pais = 'Brasil'
						 );
	
	\end{verbatim}
\end{ejemplo}


Subconsultas: 

\begin{ejemplo}
	\begin{verbatim}
		select *
		from factura
		where idCliente = (select idCliente form cliente where nombre = 'Jordi');
		
		select *
		from producto 
		where precioUnitario <= 
		(select avg(precioUnitario) from producto where idCategoria = 5)
		and idCategoria = 5;
		
		
	\end{verbatim}
\end{ejemplo}

comparando subconsultas

Subconsultas: 

\begin{ejemplo}
	\begin{verbatim}
		
		select idProducto, nombre
		from producto
		where idProducto = ANY (select idProducto from detalle_factura);
		
		
	\end{verbatim}
\end{ejemplo}

\part{Ciencia de datos}


?`Qu\'e es data ciencia?
Es el proceso de descubrir informaci\'on valiosa de los datos.

?`cu\'al es su finalidad?

\begin{enumerate}
	\item Tomar decisiones y crear estrategias de negocio. 
	\item Crear productos de software m\'as inteligentes y funcionales.
\end{enumerate}

?`De que trata este proceso?: 
\begin{enumerate}
	\item Obtenci\'on de los datos: a trav\'ez de encuestas
	\item Transformar y limpiar los datos.
	\item Explorar, analizar y visualizar datos.
	\item Usar modelos de machine learning*.
	\item Integrar datos e IA a productos de software.
\end{enumerate}




?`Qu\'e es Data Science?

Data science o ciencia de datos es el proceso de descubrir informaci\'on valiosa de los datos.

?`Cu\'al es su finalidad?
Tomar decisiones y crear estrategias de negocio
Crear productos de software m\'as inteligentes y funcionales.

?`De qu\'e trata este proceso?

Obtenci\'on de los datos
Mediciones
Encuestas
Internet

Transformar y limpiar los datos
Incompletos
Formato Incorrecto

Explorar, analizar y visualizar datos
Patrones o tendencias
Insights
Visualizaciones, gr\'aficos o reportes

Usar modelos de machine learning

Machine learning o aprendizaje autom\'atico es una rama de inteligencia artificial. Su objetivo es que las computadoras aprendan. En machine learning, las computadoras observan grandes cantidades de datos y construyen un modelo capaz de generar predicciones para resolver problemas.

Integrar datos e IA a productos de software
Ponerlos a disposici\'on del usuario final.


La ciencia de datos es una intercepci\'on de conocimiento entre (matem\'aticas y estad\'istica), (ciencias computacionales)  y conocimiento del dominio. 

\section{Proyectos data Analysis}
Poner en practica lo mas r\'apido que se pueda, tener proyectos personales, en que gasto los dineros del mes, que productos consigo cada mes, encontrar anomal\'ias, proyectos con kaggle.

\chapter{An\'alisis de datos}

?`Qu\'e es ciencia de datos y big data? ?`C\'omo afectan a mi negocio?

``?`Qu\'e haces en tu trabajo (como cient\'ifico de datos)?
Mi trabajo es crear una soluci\'on matem\'atica o estad\'istica para un problema del negocio''

\section{?`Qu\'e tipo de informaci\'on podemos analizar?}


Descubrir qu\'e tipos de informaci\'on existen, qu\'e industrias los usan y qu\'e tipo de acciones podemos tomar a partir de ellos.

Los principales datos que existen son: 

\begin{enumerate}
	\item[Personas:] Este tipo de datos lo extraemos de las personas, es decir lo generamos nosotros cuando le damos like a una foto de facebook, de preferencia, tipo de videos, de quien te gusta mas el contenido, subiendo una foto y etiquetando a un compa\~nero.
	\item[Transacciones:] las monetarias y las no monetarias, cualquier transaci\'on que hago con una tarjeta de cr\'edito o d\'ebito, cuando hacemos un pago electr\'onico o d\'igital  queda una huella, queda un registro de quien lo hizo, por que monto lo hizo y en que establecimiento lo hizo, es muy interesante por que las bancas digitales pueden hacerte recomendaciones sobre el tipo de comercio que te pod\'ia interesar.
	
	No finacieras: las compa\~nias telef\'onicas identifican cual es tu patr\'on habitual, cuantas llamadas haces, a que personas llamas, cuanto duran tus llamadas, y a partir de esto te llaman para que no abandones el servicio. 
	
	\item[Navegaci\'on web: ] Estas son las famosas cookies, ellas est\'an advirtiendo de la informaci\'on que van a recoger.
	
	\item[Machine 2 machine: ] Una conexion de una maquina a otra maquina, la usan las plataformas de transporte, google maps y para hacer la locaci\'on entre dispositivos.
	
	\item[Biom\'etricos: ] Cada vez son mas habituales y \'unicas, huellas digitales, reconocimiento facial. 

\end{enumerate}

\section{Flujo de trabajo en ciencia de datos: fases, roles y oportunidades laborales} 

Roles en datos: 

\begin{description}
	\item[Ingeniero de datos: ] crear bases de datos  Hacer que la empresa, hace la conexion de los dispositivos y las bases de datos,
	
	\item[Analista business intelligence: ] A partir de la informaci\'on que ha creado el ingeniero de datos va extraer la data, crear cuadros de control, crear dashboard, monitoreo, va automatizar estos procedimientos para que cualquier persona de la empresa pueda interpretarla, las herramientas mas utilizadas son SQL y Excel. No necesariamente sabe Python.
	
	\item[Data Scientist: ] Sabe hacer el rol del analista, sabe extraer la informaci\'on y sabe predecir, con las herramientas de estad\'istica, nos gu\'ia a donde vamos. 
	
	\item[Data Translator: ] Nos ayuda a proyectar el equipo, nos ayuda a comunicar con los otros equipos del negocio. 
	
	
\end{description}

\section{Herramientas para cada etapa del an\'alisis de datos}

El primero es el rol del analista y del ingeniero estas son las personas que crean bases de datos y utilizan SQL, se sintetiza la informaci\'on de la base de datos. 

El cient\'ifico de datos son herramientas predictivas, son R y Python, R es mas estad\'istico an\'alisis descriptivo, 

\section{Python en ciencia de Datos}

Por que numpy para el análisis de datos. Tenemos tres cosas a destacar 
\begin{enumerate}
	\item Un poderoso objeto array multidimensional.
	\item Funciones matem
\end{enumerate}

Crear un virtual environments ejecutamos la siguiente linea de comando

\begin{verbatim}
	python3 -m nombreDelProyecto ./venv
	source bin/activate
\end{verbatim}




\addcontentsline{toc}{chapter}{\numberline{}Bibliograf�a}
\begin{thebibliography}{99}
	\bibitem{Brenner2008} S.C. Brenner, L.R. Scott
	{\it The Mathematical Theory of Finite Element Methods}, $3^{\text{a}}$ ed., USA: Springer, 2008.
	\bibitem{Dietrich2007} D. Braess, {\it FINITE ELEMENTS Theory, Fast Solvers, and Applications in Elasticity Theory}, $3^{\text{a}}$ ed., USA: Cambridge University Press, 2007.
	\bibitem{Brezis} H. Brezis, {\it Functional analysis, Sobolev spaces and partial differential equations}, New York: Springer, 2011.	
	\bibitem{Gockenbach2006} M.S. Gockenbach, {\it Understanding and implementing the finite element method}, Philadelphia, PA: Society for Industrial and Applied Mathematics (SIAM), 2006.
	\bibitem{Ciarlet1978} P. G. Ciarlet, {\it The finite element method for elliptic problems}, Amsterdam - New York - Oxford: North-Holland Publishing Company, 1978.
	\bibitem{Rudiger2013} R. Verf\"urth, {\it A posteriori error estimation techniques for finite element methods}, Oxford: Oxford University Press, 2013.
\end{thebibliography}
\end{document}